\section{ACADEMIC PROJECTS}
{\customcvproject{Ph.D. Thesis | Doctoral Researcher \\ Information Retrieval (IR) \& ML for conducting Systematic Reviews}{August 2017 - September 2021}
  {\begin{itemize}
    \item My research focuses on reducing the medical cost of assessing documents by using, extending and developing information retrieval tool-kits and applying continuous active learning with supervised Machine Learning techniques in Java and Python to create intelligent digital search assistants.
    \item These assistants are developed in such way to collaborate with human reviewers to complete the tasks of creating systematic reviews for diagnostic test accuracy in IR
    \item Evaluating information retrieval search strategies for document ranking and screening models after replicating them. Performing statistical significance tests (ANOVAs, Bonferonni corrections, Paired-wise t-tests).
    \item Not graduated yet.
  \end{itemize}
  }
}

{\customcvproject{Master of Science Thesis | Postgraduate} {February 2017 – July 2017}
{\begin{itemize}
  \item The purpose of the research was to assert the generality of the predictive RFMTC model as an improved more customisable alternative of the RFM marketing model and other Machine Learning algorithms (i.e. SVM, K-Means). Justify the additional implementation complexity of some model parameters e.g. the Time since first purchase or donation of a customer in a certain period and the Churn rate as a productive one. I used extensively the R scripting language, the CRAN repository for the minimisation of objective non-linear functions and the Excel Solver Add-in.
\end{itemize}
}
}


{\customcvproject{Bachelor of Science Thesis
| Undergraduate}{January 2016 – May 2016}
{\begin{itemize}
  \item Extended a system that encourages Movie Lens users by proposing new films tailored to the needs of each user. More specifically, the SVD algorithm (http://www.timelydevelopment.com/demos/NetflixPrize.html) was parallelised in Java and Apache Spark, the third optimal algorithm in the Kaggle competition conducted by Netflix. The objective of the competition was to find the best collaborative filtering algorithm for predicting user ratings for films based on previous reviews.
\end{itemize}
}
}